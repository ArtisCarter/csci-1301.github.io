% !TeX document-id = {543bad56-0a75-4f2c-a11d-a94f42751e3d}
% !TeX TXS-program:compile = txs:///xelatex/[--shell-escape]
\documentclass[border=20pt,varwidth=\maxdimen]{standalone}
\renewcommand\familydefault{\sfdefault} % Default family: serif 
\usepackage[usenames,dvipsnames]{xcolor}
\usepackage{tikz}
%\usepackage{soul}
\usetikzlibrary{calc} 
\usetikzlibrary{arrows, decorations.markings,positioning,backgrounds,shapes}
\usetikzlibrary{patterns}
\usetikzlibrary{fit}
\usepackage[normalem]{ulem}
\usepackage{minted}

% Colors for c, assembly and machine codes.
\definecolor{c}{HTML}{002f55}
\definecolor{a}{HTML}{A5ACAF}
\definecolor{m}{HTML}{00AEEF}
% Colors for compiler and assembler.
\definecolor{compiler}{HTML}{64A0C8}
\definecolor{assembler}{HTML}{44D62C}

\begin{document}
% C Code
\newsavebox{\ccode}
\begin{lrbox}{\ccode}
\RecustomVerbatimEnvironment{Verbatim}{BVerbatim}{}%
\begin{minted}{c}
printf("Hello, ");
printf("World!");
\end{minted}
\end{lrbox}
% Python Code 
\newsavebox{\pycode}
\begin{lrbox}{\pycode}
	\RecustomVerbatimEnvironment{Verbatim}{BVerbatim}{}%
	\begin{minted}{Python}
print("Hello, ");
print("World!");
	\end{minted}
\end{lrbox}
% Python Code 
\newsavebox{\cscode}
\begin{lrbox}{\cscode}
	\RecustomVerbatimEnvironment{Verbatim}{BVerbatim}{}%
	\begin{minted}{csharp}
Console.Write("Hello, ");
Console.Write("World!");
\end{minted}
\end{lrbox}
% IL code
% Taken from https://www.developer.com/microsoft/c-sharp/c-and-intermediate-language-il/
\newsavebox{\ilcode}
\begin{lrbox}{\ilcode}
	\RecustomVerbatimEnvironment{Verbatim}{BVerbatim}{}%
\begin{minted}{text}
.maxstack  8
IL_0000:  ldstr "Hello"
IL_0005:  call  void [mscorlib]
System.Console::WriteLine(string)
\end{minted}
\end{lrbox}

% Machine code
% Extracted from what you get if you run xxd -b a.out after compiling the following source code with gcc:
%int main(void)
%{
%	int age = 10;
%	char initial = 'C';
%}
\newsavebox{\mcode}
\begin{lrbox}{\mcode}
	\RecustomVerbatimEnvironment{Verbatim}{BVerbatim}{}%
	\begin{minted}{text}
01000010 01001001
00000000 00101110
	\end{minted}
\end{lrbox}

\newsavebox{\mcodea}
\begin{lrbox}{\mcodea}
	\RecustomVerbatimEnvironment{Verbatim}{BVerbatim}{}%
	\begin{minted}{text}
01000010 01001001
	\end{minted}
\end{lrbox}

\newsavebox{\mcodeb}
\begin{lrbox}{\mcodeb}
	\RecustomVerbatimEnvironment{Verbatim}{BVerbatim}{}%
	\begin{minted}{text}
01000010 01001001
	\end{minted}
\end{lrbox}
	
\begin{tikzpicture}[
		block/.style={
				rectangle,
				thick,
				minimum width=5em,
				align=center,
				rounded corners,
				minimum height=2em
			},
		data/.style={
				cylinder,
				draw=black,
				thick,
				aspect=0.7,
				minimum height=1.7cm,
				minimum width=1.5cm,
				shape border rotate=90,
				cylinder uses custom fill,
				cylinder body fill=red!30,
				cylinder end fill=red!10
			}
	]
% Code blocks
%% High-level Language
	\node [color=c] (c) {High-Level Language};
%%% Compiled
	\node [below left = .3cm and .1cm of c, block, draw=c] (compiled) {Compiled};
	\node [below right = -.5cm and .2cm of compiled, block, fill=c!20, draw=c] (compiledcode) {\usebox\ccode};
%%% Interpreted
	\node [below  = 1cm of compiled, block, draw=c] (interpreted) {Interpreted};
	\node [below right = -.5cm and .2cm of interpreted, block, fill=c!20, draw=c] (interpretedcode) {\usebox\pycode};
%%% Managed
	\node [below = 2cm of interpreted, block, draw=c] (managed) {Managed};
	\node [below right = -.5cm and .2cm of managed, block, fill=c!20, draw=c] (managedcode) {\usebox\cscode};
%% Assembly
	\node [right = 5cm of c, color=a] (assemb) {Intermediate Language};
	\node [below = 4.5cm of assemb, block, fill=a!20, draw=a] (ilcode) {\usebox\ilcode};	
%% Machine-level Language
	\node [right = 5cm of assemb, color=m] (machine) {Machine Language};
	\node [below = .5cm of machine, block, fill=m!20, draw=m] (machinecode1) {\usebox\mcode};	
	\node [below = 2cm of machine, block, fill=m!20, draw=m] (machinecode2a) {\usebox\mcodea};	
	\node [below = 3cm of machine, block, fill=m!20, draw=m] (machinecode2b) {\usebox\mcodeb};
	\node [below = 4.5cm of machine, block, fill=m!20, draw=m] (machinecode3a) {\usebox\mcodea};	
	\node [below = 5.5cm of machine, block, fill=m!20, draw=m] (machinecode3b) {\usebox\mcodeb};	
%% Output
\node [right = 3cm of machine] (output) {Output (on screen)};
\node [below = .5cm of output, block, draw=black] (output1) {Hello, World!};	
\node [below = 2cm of output, block, draw=black] (output2a) {Hello,};	
\node [below right = 3cm and -1.7cm of output, block, draw=black] (output2b) {\phantom{Hello, }World!};
\node [below = 4.5cm of output, block, draw=black] (output3a) {Hello,};	
\node [below right = 5.5cm and -1.7cm of output, block, draw=black] (output3b) {\phantom{Hello, }World!};	
% Arrows
%% Compiler
\draw[thick, ->, color=compiler] (compiledcode)  to node[above]{Compiler} (machinecode1) ;
\draw[thick, ->, color=compiler] (managedcode)  to node[above]{Compiler} (ilcode) ;
%% Interpreter
\draw[thick, ->, color=assembler] ($(interpretedcode.east)+(0, .35)$)  to node[above]{Interpreter} (machinecode2a) ;
\draw[thick, ->, color=assembler] ($(interpretedcode.east)+(0, -.35)$)  to node[above]{Interpreter} (machinecode2b) ;
\draw[thick, ->, color=assembler] ($(ilcode.east)+(0, .55)$)  to node[above]{Interpreter} (machinecode3a) ;
\draw[thick, ->, color=assembler] ($(ilcode.east)+(0, -.5)$)  to node[above]{Interpreter} (machinecode3b) ;
% Execution
\draw[thick, ->] (machinecode1)  to node[above]{Execution} (output1);
\draw[thick, ->] (machinecode2a)  to node[above]{Execution} (output2a);
\draw[thick, ->] (machinecode2b)  to node[above]{Execution} (output2b);
\draw[thick, ->] (machinecode3a)  to node[above]{Execution} (output3a);
\draw[thick, ->] (machinecode3b)  to node[above]{Execution} (output3b);
%	\node [block, fill=yellow!20, draw=yellow] (program) at (3,1) {Program};
%	\node [block, fill=green!20, draw=green] (cli) at (3,-1) {C.L.I.};
%	\node [block, fill=green!20, draw=green] (process) at (8, -1) {Software to process queries};
%	\node [block, fill=green!20, draw=green] (access) at (8, -3) {Software to access data};
%	\node [data] (data) at (6, -6) {Data};
%	\node [data] (metadata) at (10, -6) {Catalog};
%
%	%% Frame
%
%	
%	\draw[thick,dotted] ($(cli.north west)+(-0.25,0.5)$) node[above right]{DBMS Software} rectangle ($(access.south east)+(0.75,-0.5)$) ;
%	%% Arrows
%
%	\draw [<->, thick] (user) to [bend left] (program.west);
%	\draw [<->, thick] (user) to [bend right] (cli.west);
%	\draw [<->, thick] (program) to [bend left] (process);
%	\draw [<->, thick] (cli) to [bend right] (process);
%	\draw [<->, thick] (process) to (access);
%	\draw [<->, thick] (access) to (data);
%	\draw [<->, thick] (access) to (metadata);

	%	% Caption:
	%	\node (caption) at (-5.5,2) {	Common term, \emph{CS term}, \underline{Relational Model term}, \sout{To be avoided}};
	%	
	%	% Table:
	%	\node (name) {\textbf{STUDENT}};
	%	\node [below of=name] (Name) {\textbf{Name}};
	%	\node [right= 3 of Name] (dots1) {\(\cdots\)};
	%	\node [right= 3 of dots1] (Major) {\textbf{Major}};
	%	\node [below of= Name] (Morgan) {Morgan};
	%	\node [below of= dots1] (dots2) {\(\cdots\)};
	%	\node [below of= Major] (CS) {CS};
	%	\node [below of= Morgan] (dots3) {\(\vdots\)};
	%	\node [below of= dots2] (dots4) {\(\vdots\)};
	%	\node [below of= CS] (dots5) {\(\vdots\)};
	%	\node [below of= dots3] (Bob) {Bob};
	%	\node [below of= dots4] (dots6) {\(\cdots\)};
	%	\node [below of= dots5] (IT) {IT};
	%	% Annotations:
	%%	\draw [-, dashed] (-1.5, 0.75) -- ++ (11,0);
	%%	\draw [-, dashed] (-1.5, -1.45) -- ++ (11,0);
	%	\node (meta) at (11.2, -0.35) {\(\left.\rule{0cm}{1.2cm}\right\}\) Structure, \emph{meta-data}, \underline{schema}};
	%	\node (data) at (10.4, -3.0) {\(\left.\rule{0cm}{1.65cm}\right\}\)  \emph{data}, \underline{relation state}};
	%	\begin{scope}[on background layer]
	%		\node [fit=(name)(IT), fill=gray!20, draw = none, inner sep= 15pt] (box1) {};	
	%		\node [fit=(Name)(Major), fill=blue!20, draw = none, inner sep= 5pt] (box2) {};	
	%		\node [fit=(Bob)(IT), fill=red!20, draw = none, inner sep= 10pt] (box3) {};	
	%		\node [fit=(Name)(Bob), draw = black, inner sep= 3pt] (box4) {};	
	%		\node [fit=(IT), draw = black, inner sep= 3pt] (box5) {};	
	%				
	%%		\draw [fill=gray!20, draw=none] (IT)+(1,-0.5) rectangle (name.north west);
	%%		
	%		%
	%%		\draw [fill=blue!20, draw=none] (Bob)+(0.6,-0.2) rectangle  (Name.north west);
	%%		\draw [pattern=north west lines, pattern color=blue!40, draw=none] (Bob)+(0.6,-0.2) rectangle  (Name.north west);
	%%		\draw [] (Major.south east) rectangle  (Name.north west);
	%%		\draw [] (CS.south east) rectangle  (Morgan.north west);
	%%		\draw [] (IT.south east) rectangle  (IT.north west);
	%%		\node[inner sep=10pt,draw,fit=(IT)] (box) {};
	%%		\node [fit=(Bob)(Name)(Bob)(Name), fill=blue!20] () {};
	%	\end{scope}
	%	
	%	\node (names) [left = 1cm of name] {Table name, \emph{file name}, \underline{relation name}};
	%	\draw[->] (names) -- (name);
	%	\node (Table) [above = 0.1 of name, xshift = -3.5cm]{\textcolor{gray}{Table, \emph{file}, \underline{relation}}};
	%	\draw[->, bend right=-80, gray] (Table.east) -- (box1);
	%	\node (header) [left = 1.4cm of Name] {\textcolor{blue!50}{Heading, column names, \underline{list of attributes}}};	
	%	\draw[->, blue!50] (header) -- (box2);
	%
	%%	\node (acc) [left = 0.9cm of names, yshift = -0.4cm] {Meta-data \(\left\{\rule{0cm}{0.9cm}\right.\)};	
	%	
	%	\node (entry) [left = 1.6cm of Bob] {\textcolor{red!50}{Entry, row, line, \emph{data record}, \underline{tuple}}};
	%	\draw[->, red!50] (entry) -- (box3);
	%	
	%	\node (column) [left = 1.6cm of Bob, yshift=-1cm] {Column, \sout{field}};
	%	\draw[->] (column) -| (box4);
	%	
	%	\node (value) [left = 1.6cm of Bob, yshift=-2cm] {Value, \emph{data element}, \underline{element}, \sout{field}};
	%	\draw[->] (value) -| (box5);
\end{tikzpicture}


\end{document}